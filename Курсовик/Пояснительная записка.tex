\documentclass[russian,utf8,nocolumnxxxi,nocolumnxxxii]{eskdtext}
\usepackage[T1,T2A]{fontenc}
\usepackage[utf8]{inputenc}
\usepackage[english,ukrainian,russian]{babel}
\usepackage{amsmath,amsfonts,amssymb,amsthm,mathtools} % AMS
\usepackage{icomma} % "Умная" запятая: $0,2$ --- число, $0, 2$ --- перечисление
%% Перенос знаков в формулах (по Львовскому)
%\newcommand*{\hm}[1]{#1\nobreak\discretionary{}
%{\hbox{$\mathsurround=0pt #1$}}{}}
\usepackage{cancel} % Для зачёркивания цифр
%%% Работа с картинками
\usepackage{graphicx}  % Для вставки рисунков
%\graphicspath{{images/}{images2/}}  % папки с картинками
%\setlength\fboxsep{3pt} % Отступ рамки \fbox{} от рисунка
%\setlength\fboxrule{1pt} % Толщина линий рамки \fbox{}
\usepackage{wrapfig} % Обтекание рисунков текстом
\usepackage{tikz}
\usepackage{siunitx}
\usepackage[american,cuteinductors,smartlabels]{circuitikz}
\usepackage[backend=biber]{biblatex}
\addbibresource{error_estimation_otchet.bib}
\usepackage[]{hyperref}
\hypersetup{
colorlinks=true,
}
\usepackage{textcomp}
\newcommand{\No}{\textnumero}
\ESKDdepartment{Федеральное агентство по образованию}
\ESKDcompany{Санкт-Петербургский государственный электротехнический университет "ЛЭТИ"}
\ESKDtitle{Пояснительная записка к Курсовой работе}
\ESKDsignature{Вариант N1}
\ESKDauthor{Авсюкевич С. В.}
\ESKDchecker{Прокшин~А.~Н.}
\ESKDdocName{по дисциплине "Информатика"}

\begin{document}

\begin{center}

Федеральное агентство по образованию

Санкт-Петербургский государственный

электротехнический университет "ЛЭТИ"

\end{center}

\vspace{8em}

\begin{center}
ПОЯСНИТЕЛЬНАЯ ЗАПИСКА К КУРСОВОЙ РАБОТЕ

по дисциплине "Информатика"

\end{center}

\vspace{2.5em}

\begin{center}

Вариант №1

\end{center}


\vspace{20em}

\begin{center}

2018

\end{center}
\newpage

\tableofcontents

\newpage
\section{Введение}

В настоящее время при решении различных как прикладных инженерных, так и чисто исследовательских задач, возникает необходимость в использовании широкого круга алгоритмов из множества разделов математики. Между тем самостоятельная реализация многих алгоритмов на некотором языке программирования может быть сложна и избыточна. Вследствие этого широкое распространение получили математические пакеты и системы компьютерной алгебры, такие как: MatLab, Octave, SciLab, Mathematica, Reduce, Mapple, призванные избавить пользователя от рутинных процедур, предоставить удобный интерфейс взаимодействия с уже написанным программным кодом и быстрым созданием нового. К сожалению, некоторые из перечисленных выше математических пакетов, будучи коммерческими по природе, имеют пакетом SciLab и системой компьютерной алгебры Reduce.

\newpage
 \subsection{Цель курсовой работы}
Ууметь применять персональный компьютер и математические пакеты прикладных программ в инженерной деятельности.
\subsection{Тема курсовой работы:}
Решение математических задач с использованием математического пакета «SciLab» и системы компьютерной алгебры «Reduce».

\subsection{Задание на курсавую работу}

\begin{enumerate}
\item Даны функции $f(x)=\sqrt{3}(x)+cos(x)$ и $g(x)=cos(2x+(\dfrac{\pi}{3})-1$
\begin{itemize}
  \item Решить уравнение f(x)=g(x)
  \item Исследовать функцию h(x)=f(x)-g(x) на промежутке $[0;\frac{5\pi}{6}]$
\end{itemize}
\item Найти коэффициенты кубического сплайна, интерполирующего данные, представленные в векторах $V_y$ и $V_x$.
\\Построить на графике функцию f(x), полученную после нахождения коэффициентов кубического сплайна.
\\Представить графическое изображение результатов интерполяции исходных данных различными методами с использованием встроенных функций.
\item Решить задачу оптимального распределения неоднородных ресурсов. На предприятии постоянно возникают задачи определения оптимального плана производства продукции при наличии конкретных ресурсов (сырья, полуфабрикатов, оборудования, финансов, рабочей силы и др.) или проблемы оптимизации распределения неоднородных ресурсов на производстве.
\\{\bfПостановка задачи.}  Для изготовления n видов изделий $N_1$, $N_2$, ..., $N_n$ необходимы ресурсы m видов: трудовые, материальные, финансовые и др. Известно требуемое количество отдельного i-гo ресурса для изготовления каждого j-го изделия. Назовем эту величину нормой расхода  $c_ij$. Пусть определено количество каждого вида ресурса, которым предприятие располагает в данный момент, - $a_i$. Известна прибыль $П_i$, получаемая предприятием от изготовления каждого j-го изделия. Требуется определить, какие изделия и в каком количестве должны производиться предприятием, чтобы прибыль была максимальной.
\end{enumerate}

%\begin{figure}[h]
%\begin{center}
%\includegraphics{Третье задание.jpg}}  
%\end{center}
%\end{figure}

\newpage

\section{Исследование функции}

Исследование функции — задача, заключающаяся в определении основных параметров заданной функции.

Даны функции $f(x)=\sqrt{3}sin(x)+cos(x),g(x)=cos(2x+\dfrac{\pi}{3})-1$
\begin{enumerate}
\item Решить уравнение $f(x)=g(x)$
\item Исследовать функцию $h(x)=f(x)-g(x)$ на промежутке $[0;\dfrac{5\pi}{6}]$
\end{enumerate}

\subsection{Решение уравнения}

Уравнение – это равенство, содержащее одно или несколько неизвестных, при условии, что ставится задача нахождения тех значений неизвестных, для которых оно истинно.

Решить уравнение – это значит найти все значения неизвестных, при которых оно обращается в верное числовое равенство, или установить, что таких значений нет.

Обычно при использовании мат. пакетов решение нелинейных уравнений можно получить двумя путями – численно и аналитически. Поскольку в «SciLab» с помощью стандартных функций можно получить только численное решение, при нахождении аналитического воспользуемся системой компьютерной алгебры «Reduce».

\subsection{Аналетический метод решения}

Аналетический метод решения - это решение, представленное в виде формулы (и соответственно полученное тоже путём математических выкладок).

Для отыскания аналитического решения воспользуемся функцией solve из системы компьютерной алгебры «Reduce» где:
\begin{itemize}
 \item expr – список из уравнений (то есть система)
 \item var – список из переменных, относительно которых решаются уравнения expr
\end{itemize}

При попытке разрешить уравнение h(x)= 0 относительно x:\\
solve(sqrt(3)sin(x)+cos(x)-cos(2x+pi/3)-1,x);\\
получаем:

$$
\left\{x=root_{-}of\left(cos\left(\frac{6x_{-}+\pi}{3}\right)-cos(x_{-})-sqrt(3)sin(x_{-})-1, x_{-}, tag_{-2}\right)\right\}
$$

То есть решение данного уравнения не было найдено.
Упростим данное уравнение, воспользовавшись двумя тригонометрическими тождествами: 
\begin{align}
sin(x+y)=& sin(x)cos(y)+cos(x)sin(y)\\
cos(2x)=&1-2sin^2(x) 
\end{align}

Выразим множетели функции $f(x)$ таким образом:
\begin{align*}
\sqrt3=&2 cos\frac{\pi}{6},\\   
1=&2 sin\frac{\pi}{6}
\end{align*}

Функцию $f(x)=\sqrt3 sin(x)+cos(x)$ запишем так:
\begin{align*}
sin(x) \times 2 &cos(\frac{\pi}{6})+cos(x) \times 2 sin(\frac{\pi}{6})\\
2\times(sin(x)\times &cos(\frac{\pi}{6})+cos(x) \times sin(\frac{\pi}{6}))\\
2& sin(x+\frac{\pi}{6})
\end{align*}

Функцию $g(x)=cos(2x+\frac{\pi}{3})-1$ запишим так:
\begin{align*}
\cancel{1}-2sin^2&(x+\frac{\pi}{6})-\cancel{1}\\
2sin^2&(x+\frac{\pi}{6})
\end{align*}



И получим тривиальное уравнение, эквивалентное исходному
$$2(sin(x+\frac{\pi}{6})+sin^2(x+\frac{\pi}{6}))=0$$

Применим к нему функцию solve в программе "Reduce":\\
solve(2sin(x+pi/6)*(1+sin(x+pi/6)));
и получим решение:

\begin{align*}
	x=&\frac{\pi(arbint(4)+5)}{6},& x=&\frac{\pi(12arbint(4)-1)}{6},\\ 
x=&\frac{2\pi(arbint(3)+2)}{3},& x=&\frac{2\pi(3arbint(3)-1)}{3} 
\end{align*}

где arbint (arbitrary integer) является произвольным целым числом. Запишем решение в более привычной форме:

$$x_1=\frac{5}{6}*\pi+2n\pi,n\in Z$$
$$x_2=-\frac{1}{6}*\pi+2n\pi,n\in Z$$
$$x_3=\frac{8}{6}*\pi+2n\pi,n\in Z$$
$$x_4=-\frac{4}{6}*\pi+2n\pi,n\in Z$$\\

Периодические решения для $x_3$ и $x_4$ совпадают, а периодическое решение для $x_2$ можно записать в виде:
$$x_2=\frac{11}{6}*\pi+2n\pi,n\in Z$$

Таким образов воспользовавшись математическим пакетом "Reduce" мы получили ответ Аналетическим способом. Но для более полн	 картины мы должны найти корни и Числовым методом используя пакет "Scilab".

\subsection{Числовой метод решения}

Для отыскания численного решения воспользуемся стандартной функцией «SciLab» fsolve.

Очевидно, что функция h(x), являющаяся линейной комбинацией периодических функций, будет иметь период равный наименьшему общему кратному периодов этих функций, то есть $T_h=HOK(T_f,T_g)=HOK(2\pi,\pi)=2\pi.$ Таким образом, достаточно численно отыскать корни на отрезке $[0,2\pi]$ и получить периодическое решение. 

Поскольку функция fsolve основана на методе Ньютона, требуется задать начальную точку или интервал для поиска корней. С целью отыскания начальных точек построим график функции h(x) на данном отрезке:
\begin{center}
function y=h(x)\\
y=sqrt(3)*sin(x)+cos(x)-cos(2*x+\%pi/3)+1\\
endfunction\\
plot(0:0.01:2*\%pi,h)\\
\end{center}

В результате выполнения эти команд построился график изображённый на рисунке \eqref{graf1}.
\begin{figure}[h]
\begin{center}
\begin{tikzpicture}
\begin{scope}[scale=1]

\draw[thin, ->] (-5,0) -- (5,0) node[right] {$X$};
\draw[thin, ->] (0,-3) -- (0,5) node[left] {$Y$};

%\foreach \x\xtext in{-1/-1,0.38/0.38,2/2,2.6/2.6,} %
%\draw (\x,0.1) -- (\x,-0.1) node[below] {$\xtext$}

%\draw[domain=-10:10, smooth, purple] plot ({\x},{(((sqrt(3))*(sin((\x))))+(cos((\x)))-(cos(2*(\x)+(pi\3)))+1)});
\draw[domain=-5:5, smooth, purple] plot ({\x},{(sqrt(3)*sin(\x r)+cos(\x r)-cos((2*(\x r))+(pi/3 r))+1)});
%\draw[domain=-5:5] plot ({\x},{((\x)*(\x))+2});

%\draw [domain=-10:10, smooth, green] plot ({(\x)},{sin(\x r)})


\end{scope};
\end{tikzpicture}
\caption{График функции $f(x)=\sqrt3 sin(x)+cos(x)-cos(2x+\frac{\pi}{3})+1}$
\label{graf1}
\end{center}
\end{figure}

Исходя из вида графика можно предположить о наличии трех или четырех корней (в окрестности точки x = 4.2 функция предположительно может дважды переходить через ноль).

Используя полученное знание о поведении функции воспользуемся функцией fsolve:

$[x,v] = fsolve(x0,f)$, где:

x0 – вектор начальных значений для итеративного алгоритма отыскания нулей

f – функция, для которой осуществляется поиск нулей

x – вектор нулей функции, полученных при работе алгоритма из точек x0

v – вектор значений функции в точках x

Для проверки предположения о четырех корнях укажем две начальных точки поиска с разных сторон от локального максимума, находящегося около\\ x= 4.2

Листинг кода:
\begin{center}
x0 = [3, 3.9,4.5,5.5];\\
$[x,v] = fsolve(x0,h)$\\

v =\\
-0.2220446  0. 0. 0.7771561\\
x =\\
2.6179939 4.1887902 4.1887902 5.759865
\end{center}

\begin{figure}[h]
\begin{center}
\begin{tikzpicture}
\begin{scope}[scale=1]

\draw[thin, ->] (-5,0) -- (5,0) node[right] {$X$};
\draw[thin, ->] (0,-3) -- (0,5) node[left] {$Y$};

\foreach \x\xtext in{-1/-1,0.38/0.38,2/2,2.6/2.6,} %
%\draw (\x,0.1) -- (\x,-0.1) node[below] {$\xtext$}

%\draw[domain=-10:10, smooth, purple] plot ({\x},{(((sqrt(3))*(sin((\x))))+(cos((\x)))-(cos(2*(\x)+(pi\3)))+1)});
\draw[domain=-5:5, smooth, purple] plot ({\x},{(sqrt(3)*sin(\x r)+cos(\x r)-cos((2*(\x r))+(pi/3 r))+1)});
%\draw[domain=-5:5] plot ({\x},{((\x)*(\x))+2});

%\draw [domain=-10:10, smooth, green] plot ({(\x)},{sin(\x r)}) 455


\end{scope};
\end{tikzpicture}
\caption{График функции $f(x)=\sqrt3 sin(x)+cos(x)-cos(2x+\frac{\pi}{3})+1}$
\label{graf1}
\end{center}
\end{figure}

\end{document}