\documentclass[a4paper,12pt]{article}
\usepackage[T2A]{fontenc}
\usepackage[utf8]{inputenc}
\usepackage[english, russian] {babel}
\usepackage{tikz}
\usepackage[european,cuteinductors, smartlabels]{circuitikz}
\title{Практическая работа 2}

\author{студент Авсюкевич Станислав Витальевич}


\begin{document}
\maketitle

\section{Практическая работа №2}
\subsection{Основное задание}
Построение ковариантной и конравариантной координаты вектора в косоугольной системе координат.

Необходимо построить косоугольную ось координат, вектор в этой оси, построить орты этого вектора, изобразить сложение векторов создающий вектор $\vec{х}$, указать проекцию вектора на косоугольную ось координат под $90^0$. Построить дополнительную ось координат перпендикулярную вектору $\vec{х}$, и спроецировать на неё точки проекции.

В данной практической работе мне заданы следующие параметры косоугольной системы координат и вектора в ней:
\begin{enumerate}
	\item Угол косоугольной системы координат $t^0=65^0$
	\item Угол вектора от оси Х равный $х=35^0$
\end{enumerate}


\begin{tikzpicture}
\newcommand{\alfa } {35}
\newcommand{\teta } {65}
\newcommand {\D } {7}

\draw(0,0) node [below] {0};%Ставлю начало координат.
\draw[thin,->,>=stealth'] (0,0) -- (9,0) node[below] {$X$}; %Координата ОХ
\draw[thin,->,>=stealth'] (0,0) -- ({9*cos(\teta)},{9*sin(\teta) }) node[left] {$Y$}; %Координата ОY
\draw[thin,->,>=stealth'] (0,0) -- ({\D*cos(\alfa) },{\D*sin(\alfa) }) node[at end, above right] {$\vec{x}$}; %Вектор Х
\draw[thin,dashed] ({\D*cos(\alfa) },{\D*sin(\alfa) }) -- ({\D*cos(\alfa) },0)node[at end,below] {$x_1$};  %JОпускаю перпендикуляр к оси ОХ
\draw[thin]({\D*cos(\alfa) },{\D*sin(\alfa) }) -- ({\D*sin(\alfa )},0) node[at end,below] {$x^1$};
\draw[thin,dashed,->] (0,0) -- (0,{9*sin(90)}) node [at end,left] {$Y_0$}; %Создание прямой координатной оси ОY_0.
\draw[thin, dashed]({\D*cos(\alfa)},{\D*sin(\alfa )}) -- (0,{\D*sin(\teta )}) node[at end, left] {$x_2$}; %Строю отрезок к оси ОY_0.
\draw[thin]({\D*cos(\alfa) },{\D*sin(\alfa) }) -- (0,{\D*sin(\alfa) }) node[at end,above left] {$x^2$}; %Строю перпендикулярный отрезок к оси OY_0.
\draw[thin,<->] (1.5,0) arc(0:35:1.5) node[midway,right] {$\alpha=35^\circ$}; %Обозначение угла альфа.
\draw[thin,<->] (8,0) arc(0:65:8) node[midway,right,above right] {$\Theta=65^\circ$}; %Обозначение угла тета.

\end{tikzpicture}
\end{document}