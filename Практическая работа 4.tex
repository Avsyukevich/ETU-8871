\documentclass[a4paper,12pt]{article}
\usepackage[T2A]{fontenc}
\usepackage[utf8]{inputenc}
\usepackage[english, russian] {babel}
\usepackage{tikz}
\usepackage[european,cuteinductors, smartlabels]{circuitikz}
\usepackage{indentfirst} % Красная строка
\title{Практическая работа 4}

\author{студент Авсюкевич Станислав Витальевич}


\begin{document}
\maketitle

\section{Практическая работа №4}

Построить молекулу метана в графическом виде. При условии что угол поворота оси OY составляет $\alpha=\frac{2\Pi}{3}$, а угол поворота оси OZ составляет $\beta=\frac{5\Pi}{6}$.

\begin{figure} [h]
\begin{center}

\begin{tikzpicture}
\begin{scope}[xscale=1,yscale=1]

		\draw [rounded corners=4pt,color=white,ball color=red,smooth] (2.5980762,-1.5)
circle (2) node {2}; 	%2 Красный шар
	
	
	\draw [rounded corners=4pt,color=white,ball color=blue,smooth] (5.1961524,- 3)
circle (2) node {3}; 	%3 Синий шар	

		\draw [rounded corners=4pt,color=white,ball color=black,smooth] (1.9857038,- 2.5606602)
circle (2.5) node {5}; 	% Чёрный шар

	\draw [rounded corners=4pt,color=white,ball color=green,smooth] (0.1485865,- 5.7426407)
circle (2) node {4}; 	%2 Зелёный шар

	\draw [rounded corners=4pt,color=white,ball color=gray,smooth] (0,0)
circle (2) node {1}; %1 Серый, первый шар.
	
	
\end{scope}
\end{tikzpicture}
\end{center}
\end{figure}


\end{document}